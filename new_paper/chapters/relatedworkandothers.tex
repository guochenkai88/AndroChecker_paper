\section{Related Work}
\textbf{Event-based system analysis}
One problem in event-based system such as Android is that the program execution is nondeterministic. Various existing works focus on addressing the challenges raised by such nondeterminism. \cite{new2015anomalies} proposes a static analysis approach called \textit{DEvA} to detect the event anomalies. The analysis process adopts several system representations, such as \textit{CFG}, \textit{ICFG} (inter-procedural \textit{CFG}) and \textit{PDG} (Program Dependence Graph). \cite{new2013identifying} presents \textit{Eos}, a static analysis technique that identifies message information for distributed event-based system and exploits data dependency analysis to produce \textit{MFG}(message flow graph). \cite{new2013p} provides a safe language \textit{P} for event-driven asynchronous systems and presents the interaction system via state machines. Compared with these event-based works, our work aims at constructing a generic model for better understanding of target Android app.

%cite{yang2015static} fundamentally proposes a program representation built by context-sensitive static analysis of callback methods to model the app's GUI. cite{kim2014fepma} provides an event-driven analysis manner to measure the power dissipation.  cite{chen2013contextual} centres around so-called Permission Event Graph built with static analysis and uses model checking to illegal interaction between an application and the Android event system. Again, cite{jensen2013automated} handle event-driven analysis to reach challenging source code for application testing. cite{machiry2013dynodroid} presents Dynodroid to monitor the reaction of an app upon each event either from system or user. cite{hsiao2014race}presents a race detection tool named CAFA for event-driven mobile systems. Although prior works present different insights of event-driven analysis in Android system, there are few efforts concerning the security impact raised by public event abuse. Our work fills this gap via systematic study of Android system design and statical analysis for a large scale of real-world apps. 
\textbf{Path-sensitive static Analysis}
Path-sensitive analysis (\textit{PSA}) benefits the program verification accuracy for they are able to reason about branch correlations.
\cite{new2013path3} exploits \textit{PSA} to detect remote code execution vulnerability on web application; while \cite{new2003path2} and \cite{new2005path6} use it to detect memory access errors. \cite{new2007path7} presents \textit{CHRONICLER}, an analyse tool that can infer accurate function precedence via inter-procedural \textit{PSA}. \cite{new2012path4} provides a path-sensitive static analyzer \textit{PAGAI} that computes inductive invariants on the numerical variables of the analyzed program.
Other works focus on improving path-sensitive analysis techniques. \cite{new2002path1} implements a verification tool \textit{ESP} that can engage \textit{PSA} in polynomial time. \cite{new2008path5} distinguishes observable and unobservable variables in path-sensitive program. Differing from previous works, we adopt \textit{PSA} to present a fine-grained callback invocation in produced model.

\textbf{Flow analyses and modelling for Android}
Program control/data flow analysis is widely used in areas of Android testing\cite{new2012concolic}\cite{new2013automated} and security \cite{new2014flowdroid}\cite{new2015DroidSafe}. Further, some prior works concern the impacts of ICC(inter-component communication) mechanism \cite{new2013epicc}\cite{new2015iccta}. These approaches aim at improving the precision via fine-grained flow- or context-sensitive static analysis.
Since Android contains a vast scale of SDKs(software development kit) and frameworks with complex inner mechanisms, recent works tend to abstract and construct a valid model to simplify analysis. \cite{new2015static} proposes representation \textit{CCFG} (callback control flow graph) for Android app, which is built using context-sensitive static analysis. \cite{new2015window} proposes the \textit{WTG}(window transition graph) to handle the possible \textit{GUI} window sequences, and develops algorithms for \textit{WTG} construction. \cite{new2013contextual} constructs \textit{PEG} (permission event graph) and checks malicious interaction via model checking. \cite{new2015jumping} abstracts useful control flow model in app to handle valid event orderings. Yet, these modelling efforts neither consider the impacts of flow condition, nor support fined-grained callbacks like system-driven callbacks and the callbacks from service. Our approach aims to construct a generic, path-sensitive and fine-grained callback model. 






%The exploration and exploitation of weaknesses within either Android system or apps are intensively focused by current related researchers. The study of memory side-channel analysis and exploitationcite{zhou2013identity, chen2014peeking, jana2012memento}uncovers an indirect attack way aiming to the weak architecture of Android public data exposure.
%Several recent researches focus on the vulnerabilities raised by Android advertising librariescite{soteris2016free}cite{sooel2016mob_ads}. Besides that, cite{zhang2016life} proposes data residue problem during app uninstallation. As a direct interaction channel with users, the Android UI also suffers from severity security threats, which is carefully discussed in cite{ren2015towards}cite{bianchi2015app}cite{akhawe2014clickjacking}cite{luo2012touchjacking}cite{huang2012clickjacking}cite{roesner2013securing}cite{lin2014screenmilker}. 
%Again,in the work of cite{mylonas2013smartphone}cite{xu2012taplogger}cite{miluzzo2012tapprints}, the sensor data could also be regarded as a sort of attack resources predicting some isolated private information including user input,fingerprints. In addtion, cite{armando2012dos} introduces a denial-of-service attack approach utilizing the AMS design defect. cite{xing2014upgrading} and cite{bugiel2012towards} present detailed studies about privilege escalation for malware through defects of Android os.  As to Android permission mechanism, cite{felt2011android}cite{au2012pscout}cite{zhang2013vetting}cite{fang2014permission} reveal security threats arised by the abuse of permission requests and propose corresponding migration strategy.
%In contrast to existing work, this paper represents the impact of public event defect of Android that affects a much wider range of system components and services. The attacks relying on such public event callbacks are firstly studied systematically.
%
%Android Vulnerability Detection
%As a critical security concerning on Android system, privacy data leakage incurs widely related research efforts. 
%Of all the detection works, dynamic taint tracking cite{enck2014taintdroid}cite{klieber2014android}cite{rastogi2013appsplayground}cite{poeplau2014execute} receives much eyesights for its sound accuracy.
%Another important complement for dynamic approach relies on static analysiscite{lu2012chex}cite{arzt2014flowdroid}cite{gordon2015DroidSafe}, which can effectively reduce the false negative rate. Further works concern the impacts of ICC(inter-component communication) mechanismcite{cao2015edgeminer}cite{octeau2013epicc}cite{li2015iccta}. In addition,
%cite{cox2014spandex} measures implicit leak of user password using symbolic execution. Our work is devoted to uncover the public event threats within real market app, which can be easily exploited by attackers to build a wide spectrum of new malwares. 




% An example of a floating figure using the graphicx package.
% Note that \label must occur AFTER (or within) \caption.
% For figures, \caption should occur after the \includegraphics.
% Note that IEEEtran v1.7 and later has special internal code that
% is designed to preserve the operation of \label within \caption
% even when the captionsoff option is in effect. However, because
% of issues like this, it may be the safest practice to put all your
% \label just after \caption rather than within \caption{}.
%
% Reminder: the "draftcls" or "draftclsnofoot", not "draft", class
% option should be used if it is desired that the figures are to be
% displayed while in draft mode.
%
%\begin{figure}[!t]
%\centering
%\includegraphics[width=2.5in]{myfigure}
% where an .eps filename suffix will be assumed under latex, 
% and a .pdf suffix will be assumed for pdflatex; or what has been declared
% via \DeclareGraphicsExtensions.
%\caption{Simulation Results}
%\label{fig_sim}
%\end{figure}

% Note that IEEE typically puts floats only at the top, even when this
% results in a large percentage of a column being occupied by floats.


% An example of a double column floating figure using two subfigures.
% (The subfig.sty package must be loaded for this to work.)
% The subfigure \label commands are set within each subfloat command, the
% \label for the overall figure must come after \caption.
% \hfil must be used as a separator to get equal spacing.
% The subfigure.sty package works much the same way, except \subfigure is
% used instead of \subfloat.
%
%\begin{figure*}[!t]
%\centerline{\subfloat[Case I]\includegraphics[width=2.5in]{subfigcase1}%
%\label{fig_first_case}}
%\hfil
%\subfloat[Case II]{\includegraphics[width=2.5in]{subfigcase2}%
%\label{fig_second_case}}}
%\caption{Simulation results}
%\label{fig_sim}
%\end{figure*}
%
% Note that often IEEE papers with subfigures do not employ subfigure
% captions (using the optional argument to \subfloat), but instead will
% reference/describe all of them (a), (b), etc., within the main caption.


% An example of a floating table. Note that, for IEEE style tables, the 
% \caption command should come BEFORE the table. Table text will default to
% \footnotesize as IEEE normally uses this smaller font for tables.
% The \label must come after \caption as always.
%
%\begin{table}[!t]
%% increase table row spacing, adjust to taste
%\renewcommand{\arraystretch}{1.3}
% if using array.sty, it might be a good idea to tweak the value of
% \extrarowheight as needed to properly center the text within the cells
%\caption{An Example of a Table}
%\label{table_example}
%\centering
%% Some packages, such as MDW tools, offer better commands for making tables
%% than the plain LaTeX2e tabular which is used here.
%\begin{tabular}{|c||c|}
%\hline
%One & Two\\
%\hline
%Three & Four\\
%\hline
%\end{tabular}
%\end{table}


\section{Conclusion}
Existing work targeting at constructing control-flow based model in Android provides limited benefit when applied to real system. We proposed GPC, a generic representation with fine-grained path information, and developed algorithms for its construction and traversal. We described our proof-of-concept builder AndroChecker and presented the evaluation results on real apps. The results showed that AndroChecker can efficiently construct the expected GPC model in a fully automated way. 

%In this work, we start the first exploration stage to conduct callback logic verification. Our conceptual contribution is the Callback-based Model Checking(CMC), a novel, callback logic directed approach that using model checking to automatically verify the correctness and effectiveness of target system. We devised a new static analysis procedure for generating the Callback Transition Graph(CTG) from Dalvik bytecode over components and events of entire types. To validate the proposed analysis approach, we implemented a prototype system AndroCG, which is used to construct the CTGs for over 2000 common-used real-world apps and verify corresponding logic properties. The evaluation results demonstrate the ** and ** of our approach. 
%
%Our work leads to several questions, of which one is state exploration, the other is properties need to be extended to cater more verification requirements.
%
%There are amounts of avenues of future work, and we have already started pursuing some. A logical next step would be trying to adapt CMC to verify properties relevant to device run-time feature, such as measuring of energy and memory. For example, we could collect the runtime energy consumption during each callback transition and verify whether over-consumption exists in specified callback sequences.(e.g., using CMC verify properties relevant to device run-time feature, extend model form to verify richer features)

%\section*{Acknowledgment}
%%We would like to thank our anonymous reviewer for their insightful comments.
%This research was supported in part by National Natural Science Foundation of China (No. 61402264).
