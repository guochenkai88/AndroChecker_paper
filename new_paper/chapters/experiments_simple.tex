\section{Empirical Evaluation}
We implemented the method in a tool -- AndroChecker, which is built based on AndroGuard~\cite{new2013androguard} framework, and consists of 4,400 lines of Python code. The input of our tool is the \textit{apk} file. The output is the set of nodes and edges, path condition set, the nodes of parallel slice and related statistical results.
%In this section, we present the empirical evaluation of our approach. First we present
We perform evaluation on 20 real-world apps downloaded from in Google Play, which are used in prior works~\cite{new2015static, new2015window}. %Considering AndroChecker is able to directly analyze the real-world apps, 
%We download these apps in Google Play, which is quite different from the prior open source version. 
Then, we study three typical cases to illustrate results of \textit{PSI} in details. 

\begin{table*}[!t]
\centering
\begin{threeparttable}[b]
\caption{Evaluation results for GPC model construction algorithms.} 
\newcommand{\tabincell}[2]{\begin{tabular}{@{}#1@{}}#2\end{tabular}}
\footnotesize

\begin{tabular}{|c|c|c|c|c|c|c|c|c|c|c|c|c|c|c|c|c|c|}
\hline 
\multirow{2}{*}{Name} &\multicolumn{3}{|c|}{Application} & \multicolumn{5}{c|}{PII} & \multicolumn{3}{c|}{CBC} & \multicolumn{3}{c|}{PSI}  &\multicolumn{2}{c|}{Time(sec)}\\
\cline{2-17}
& A & S & B &      $N_{l}$ & $E_{l}$ & $hN$ &  $N_{n}$ & $E_{r}$&     $E_{j-a}$ & $E_{j-s}$ & NSlc&   N & E & Cons&  PT & MT \\ 
\hline 
\hline 
APV & 4 & 0 & 0 &    26& 41 & 18 & 31 & 62       &9 & 0 &0 &    57 & 112 & 6     & 9 & 14\\ 
\hline 
Astrid &  57 & 16 & 27 &    304& 400 & 329 & 118 & 236      & 8 & 0 &0 &    422 & 725 & 15    & 117 & 125\\ 
\hline 
BarcodeScanner & 9 & 0 & 1 &    50& 84 & 44 & 16 & 32       &4 & 0 &0 &    66 & 120 & 3    & 18 & 38\\ 
\hline 
Beem &  13 &1 & 7 &    97& 193 & 49 & 28 & 56       &18 & 2 &- &    125 & 269 & 4    & 43 & 63\\ 

\hline 
ConnectBot & 11 & 1 & 1 &    76& 114 & 51 & 49 & 98       &9 & 1 &- &    125 & 222 & 17    & 47 & 62\\ 
\hline 
FBReader & 9 & 9 & 9 &    86& 110 & 52 & 29 & 58       &1 & 2 &- &    115 & 171 & 12    & 128 & 141\\ 
\hline 
K9 & 32 & 9 & 8 &    196& 246 & 177 & 154 & 308       &24 & 4 &- &    350 & 582 & 11    & 66 & 97\\ 
\hline 
KeePassDroid & 24 & 1 & 2 &    115& 144 & 138 & 36 & 72       &6 & 2 &4 &    151 & 224 & 16    & 11 & 18\\ 
\hline 
Mileage & 19 & 13 & 0 &    176& 209 & 98 & 79 & 158       &3 & 8 &40 &    255 & 378 & 11    & 332 & 236\\ 
\hline 
MyTracks & 7 & 1 & 1 &    42& 64 & 37 & 41 & 82       &5 & 2 &9 &    83 & 153 & 3    & 23 & 35\\ 
\hline 
NotePad & 1 & 0 & 0 &    6& 10 & 4 & 1 & 2       &0 & 0 &0 &    6 & 12 & 0    & 19 & 23\\ 
\hline 
NPR & 31 & 13 & 51 &    235& 370 & 153 & 217 & 434       &10 & 19 &- &    452 & 827 & 32    & 124 & 139\\ 
\hline 
OpenManager & 53 & 16 & 5 &    321& 596 & 272 & 272 & 544       &75 & 8 &- &    593 & 1223 & 1    & 119 & 121\\ 
\hline 
OpenSudoku & 10 & 0 & 0 &    62& 88 & 49 & 39 & 78       &14 & 0 &0 &    101 & 180 & 6    & 3 & 5\\ 
\hline 
SipDroid & 21 & 3 & 16 &    127& 203 & 106 & 48 & 96       &14 & 7 &16 &    175 & 312 & 10    & 15 & 26\\ 
\hline 
SuperGenPass & 2 & 0 & 1 &    11& 19 & 10 & 18 & 36       &1 & 0 &0 &    29 & 56 & 0    & 5 & 6\\ 
\hline 
TippyTipper & 83 & 11 & 102 &    536& 848 & 359 & 373 & 746       &362 & 127 &817 &    909 & 2083 & 58    & 251 & 376\\ 
\hline 
VLC & 24 & 11 & 11 &    189& 248 & 112 & 211 & 422       &23 & 6 &15 &    400 & 699 & 10    & 99 & 113\\ 
\hline 
VuDroid & 5 & 0 & 0 &    22& 30 & 30 & 4 & 8       &0 & 0 &0 &    26 & 38 & 0    & 1 & 2\\ 
\hline 
XBMC & 3 & 0 & 0 &    11& 15 & 19 & 0 & 0       &2 & 0 &0 &    11 & 17 & 0    & 0.1 & 0.1\\ 
\hline
\end{tabular}
%\begin{tablenotes}
    %\item [1] 
  % \end{tablenotes}
\label{tb2: experiment1}  
\end{threeparttable}
\end{table*}


% Our system is available in website.
\subsection{Construction GPC model}
Table~\ref{tb2: experiment1} shows the evaluation results for each stage of analysis. Column ``Application'' shows the component sizes, where ``A", ``S" and ``B" represent the number of activity, service and broadcast receiver, respectively. %As described in Section \ref{background}, broadcast receiver is regarded as non-lifecycle callback listener. Therefore its callbacks are added into $N_{n}$, and related inner-component edges are added into $E_{r}$.
%
Column \textit{PII} shows the results of the inner component model, including: lifecycle nodes $N_{l}$, lifecycle edges $E_{l}$, hidden nodes $hN$, non-lifecycle nodes $N_{n}$ and register edges $E_{r}$. %On the whole, the complexity of \textit{PII} is closely related to the component size. The number of edges are more than corresponding nodes because each node at least has a pair of incoming and outgoing, except the start and terminal nodes. Normally, the number of $E_{r}$ should be twice as many as $N_{n}$, because all the $N_{n}$ are located between \textit{onActiveStart} and \textit{onActiveEnd} with two new edges added. This relationship is based on the assumption that each non-lifecycle callbacks are invoked in the \textit{active area}, and this assumption simplifies the model complexity. Another relationship between component sizes and lifecycle nodes is reflected in the formula $N_{l} + hN = A*10 + S*6 $ or $ N_{l} + hN = A*10 + S*7$. The number of lifecycle callbacks for a component (activity and service) is fixed. For activity, the number is 10, which includes 7 regular callbacks and 3 auxiliary callbacks. For service, the number in binding mode is 7 (4 regular callbacks and 3 auxiliary callbacks), and the number in starting mode is 6 (3 regular callbacks and 3 auxiliary callbacks). However, since some callbacks are invoked along with the \texttt{onCreate} (e.g., onCreateDialog which is adopted in the motivation example), we treat them as \texttt{onCreate} lifecycle nodes. As a result, the $N_{l} + hN$ are more than $A*10 + S*7$ in some apps, e.g., \textit{APV}.
%
Column \textit{CBC} shows the results of the inter component connections, including: the number of inter-component edges between activities $E_{j-a}$, the number of inter-component edges involving service $E_{j-s}$ and the sum of parallel slice nodes $NSlc$. %Normally, each activity and service should be launched through other component. Therefore the number of $E_{j-a}$ and $E_{j-s}$ should be similar to \textit{A} and \textit{S} respectively. However, the $E_{j-a}$ in some apps (e.g., \textit{Astrid} and \textit{NPR}) has a large gap to the expectation in Table \ref{tb2: experiment1}. This is caused by three reasons: \textcircled{1} activities launched by other apps can not be used to generate edges. \textcircled{2} activities and services launched by implicit intents can not be recognized. \textcircled{3} context-sensitive structure can not be handled, e.g., a loop structure launches multiple services. For $NSlc$ column, "-" represents that the parallel slice can not be counted; it exists in the scenario that a service is not terminated by the activity that launches it.
%
Column \textit{PSI} shows the result of path-sensitive conditions. We present the total nodes $N$ and total edges $E$ in this stage. From Table \ref{tb2: experiment1}, two formulas can be observed, which are $N = N_{l} + N_{n}$ and $E = E_{l} + E_{r} + E_{j-a} + E_{j-s}$. Column $Cons$ represents the path sensitive condition number of callback flow in each app. As mentioned earlier, currently AndroChecker only supports inner-procedural conditions traversal. As a result, the conditions residing out of the callback holder method would not be analyzed. 

We omit the results of marked edges (mentioned in Section \ref{jumpConfusion}) since they are the same as $E_{j-a}$. Column \textit{Time} shows the time cost of our approach. It includes two parts: parsing time (\textit{PT}) and modelling time (\textit{MT}). Since AndroChecker is to handle \textit{apk} file, the process of reverse engineering consumes significant time cost. Yet this part of work is done by AndroGuard, which is not our contribution. On the whole, the number of \textit{PT} and \textit{MT} are related to the components' size. 

\begin{table*}[!t]
\centering
\begin{threeparttable}[b]
\caption{Example path sensitive conditions in studied cases. } 
\newcommand{\tabincell}[2]{\begin{tabular}{@{}#1@{}}#2\end{tabular}}
\footnotesize

\begin{tabular}{|c|c|c|c|c|}
\hline 
Name &  PSC & Num & TranToJAVA & FP\\
\hline 
\hline 
\multirow{6}{*}{Beem }& \multirow{2}{6.3cm}{if-eq\{SharedPreferences;$\rightarrow$getBoolean (String;Z)Z, 
String;$\rightarrow$equals(PARA;)Z\} } & \multirow{2}{*}{2} & \multirow{2}{4cm}{if(SharedPreferences.getBoolean(String b, True) \&\& b.equals(PARA))} & \multirow{2}{*}{0}\\
&&&&\\
\cline{2-5}
 & \multirow{2}{6.3cm}{if-eq\{AsyncTask;$\rightarrow$getStatus()AsyncTask \$Status;,GLOBAL\}}  &\multirow{2}{*}{1} & \multirow{2}{*}{if(AsyncTask.getStatus==GLOBAL)}& \multirow{2}{*}{0}\\

&&&&\\
 \cline{2-5}
  & \multirow{2}{6.3cm}{if-eq\{ChangeStatus;$\rightarrow$access\$1400(ChangeStatus;) Landroid/widget/Button;\}}  & \multirow{2}{*}{1} & \multirow{2}{*}{if(SharedPreferences.getBoolean()}& \multirow{2}{*}{1}\\
  &&&&\\
\hline 
\multirow{2}{*}{MyTracks}  &if-eq\{Utilities;$\rightarrow$isGpsOn(Context;)Z\} & 2 &if(Utilities.isGpsOn) & 1\\
\cline{2-5} 
 &if-ne\{GooglePlayServicesUtil;$\rightarrow$a(Context; I)String;\} & 1 & if(GooglePlayServicesUtil.a) & 0\\
\hline 
\multirow{3}{*}{OpenSudoku}  & if-ne\{InputMethod;$\rightarrow$isInputMethodViewCreated()Z'\}  & 1& if(!InputMethod.isInputMethodViewCreated())& 0\\
\cline{2-5}
  & if-eq \{File;$\rightarrow$isDirectory()Z, File;$\rightarrow$isFile()Z\} &1& if(File.isDirectory() \&\& File.isFile()) & 1\\
\cline{2-5}
   & if-ne \{Iterator;$\rightarrow$hasNext()Z\}& 4& if(!Iterator.hasNext())  & 0\\
\hline 
\end{tabular}
%\begin{tablenotes}
    %\item [1] 
 %  \end{tablenotes}
\label{tb2: experiment2}  
\end{threeparttable}
\end{table*}

\subsection{Path Sensitive Conditions (PSC) Discussions}

For further understanding how our modelling supports the \textit{PSC} checking, we study the \textit{PSC} results on three typical apps combined with manual analysis. Table \ref{tb2: experiment2} shows the detailed results, where the column \textit{PSC} represents the simplified \textit{PSC} results; column \textit{Num} shows the times a \textit{PSC} appears;  column \textit{TranToJAVA} presents the translated JAVA code from the \textit{PSC} results; and column \textit{FP} shows the false positive number in \textit{Num}. Due to the limitation of space, we combine the variable defining and using together in \textit{TranToJAVA}. E.g., for \texttt{Object c ; b==c;}, we simplified as \texttt{b==Object c;}. 

\subsubsection{Beem}
\textit{Beem} provides a full featured and easy to use XMPP client on Android. As shown in Table \ref{tb2: experiment1}, app \textit{Beem} contains four \textit{PSC}s. Three of them are designed to constrain unregister actions (Definition 2), and the last one is used to constrain jumping actions (Definition 2). The unregister and jumping actions include \texttt{Button.setOnClickListener}, \texttt{View.setOnCreateContextMenuListener} and \texttt{Context.bindService}. To verify the results in Table \ref{tb2: experiment2}, we conduct a manual analysis.  After that, the real conditions can be concluded as follows: \textcircled{1} \texttt{if ( SharedPreferences paramSharedPreferences.getBoolean("use\_auto\_awa}  \\
\texttt{y", false) \&\& "use\_auto\_away".equals(paramStrin}\\
\texttt{g) )}; \textcircled{2} \texttt{if(LoginAnim.this.mTask.getStatus()==Async}\\
\texttt{Task.Status.PENDING)}; \textcircled{3} \textit{FP}. For  \textcircled{1} and \textcircled{2}, the conditions are abstracted successfully. As for \textcircled{3}, the jumping action \texttt{bindService} is directly contained by \texttt{onResume}, which indicates \textcircled{3} is an \textit{FP}. The reason is that the back traversal of AndroChecker can not correctly identify the loop structure. In detail, AndroChecker would treat \texttt{while()\{\} bindService;} as \texttt{while()\{bindService;\}}. Currently, in the byte codes generated by AndroGuard, distinguishing the two structures is intractable. This is also a critical improvement of future direction.

\subsubsection{MyTracks}
\textit{MyTracks} helps travellers to look for a way to keep track of places that they have already been to. Three \textit{PSC}s are detected in \textit{MyTracks}.  All the \textit{PSC}s are designed to constrain register actions. The register actions include \texttt{LocationManager.requestLocationUpdates } and \texttt{Button.setOnClickListener}. By manual analysis, the real \textit{PSC}s can be abstracted as  \textcircled{1} \texttt{if(Utilities.isGpsOn(getApplicationContext()))}, \textcircled{2} \textit{FP}, \textcircled{3} \texttt{if (GooglePlayServicesUtil.a(localContext,}\\
 \texttt{i).str1!=null)}. For  \textcircled{1} the real \textit{PSC} is successfully matched as Table \ref{tb2: experiment2}. For \textcircled{2} the register action is directly contained by \textit{onCreate}, and the reason of \textit{FP} is the same as \textit{Beem}. For \textcircled{3}, the PSC abstracted by our approach lacks the variable \textit{str1}. This is because \textit{str1} is an attribute of \textit{GooglePlayServicesUtil.a},  which can not be recognized by AndroChecker.

\subsubsection{OpenSudoku}
\textit{OpenSudoku} is an open source sudoku game. AndroChecker detects six \textit{PSC}s in \textit{OpenSudoku}. Five of them are designed to constrain register actions, and the last one is designed to constrain jumping actions. The register and jumping actions contain \texttt{Button.setOnClickListener} and \texttt{Context.startActivity}. Through manual analysis, the real \textit{PSC}s can be abstracted as  \textcircled{1} \texttt{if(!(InputMethod).isInputMethodViewCreated())}, \textcircled{2} \textit{FP}, \textcircled{3} \texttt{if(!Iterator.hasNext())}.
The \textit{PSC}  \textcircled{1} and \textcircled{3} are successfully detected by AndroChecker, and they have the same semantic with the corresponding items in Table \ref{tb2: experiment2}. As \textit{MyTracks} and \textit{Beem}, the \textit{FP} occurs because the loop obstacle.


%Note that the $!$ in 1) and 3) is reflected in the Table \ref{tb2: experiment2}, because it is related to the operator of byte code. 



 


%In its ten activities, seven of them contains register actions, and six of them contains jumping actions.  As shown in Table \ref{tb2: experiment1}, AndroChecker checks six valid path sensitive conditions. Four \textit{PSC}s are contained in class \texttt{IMPopupDialog}, and one condition is in class \texttt{IMControlPanelActivity}. The register actions of these five are \texttt{Button.setOnClickListener}s. The rest one is in \texttt{FileListActivity\$3}, whose register action is inter component jumping \texttt{startActivity}. Table ** shows the typical example for one of the \texttt{setOnClickListener}. The shown condition means the register action is under the condition of "if "
  





